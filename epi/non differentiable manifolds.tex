\documentclass[../main]{subfiles}
\begin{document}
\section{Non-Differentiable Manifolds}
The theory of real vector bundles is ideally suited to the study of smooth manifolds, just as the theory of complex vector bundles is suited to complex manifolds. If we are given some different category of manifolds, then it is useful to look for an appropriate corresponding type of bundle. Consider for example the category of all piecewise linear manifolds\index{piecewise linear manifold} and piecewise linear mappings. An appropriate type of bundle for this category can be described as follows. Let $B$ be a locally finite simplicial complex.\index{simplicial complex}

\begin{definition}
A \defemph{piecewise linear $\bR^n$-bundle}\index{piecewise linear}\index{piecewise linear bundle} over $B$ consists of a simplicial complex $E$ and a piecewise linear map $p:E \varrightarrow{} B$ satisfying the following local triviality\index{local triviality} condition. Each point of $B$ must possess an open neighborhood $U$ so that $p^{-1}(U)$ is piecewise linearly homeomorphic to $U \times \bR^n$ under a homeomorphism which is compatible with the projection map to $U$. (Here the open subset $U$ has the structure of a simplicial complex by Runge's theorem. See \cite{alexandroff}.)
\end{definition}

The \defemph{piecewise linear tangent bundle}\index{tangent bundle $\tangentbundle{M}$} of a piecewise linear $n$-manifold $M$ can be constructed as follows. According to B. Mazur (unfortunately unpublished) there exists a neighborhood $E$ of the diagonal in $M \times M$ so that the projection $(x,y) \mapsto x$ from $E$ to $M$ constitutes a piecewise linear $\bR^n$-bundle. Furthermore this bundle is unique up to isomorphism. (For the analogous theorem in the topological category see \cite{kirster}. Without using Mazur's theorem, one could base this discussion on the slightly more esoteric notion of a piecewise linear microbundle\index{microbundle}. See \cite{milnor1965}.)

Piecewise linear $\bR^n$ bundles over $B$ are classified by mappings of the base space $B$ into certain ``universal base space'' or ``classifying space,''\index{classifying space} which is called $\operatorname{B}(\PL_n)$. Thus the theory of characteristic classes for piecewise linear manifolds coincides with the computation of $\homology^\ast \operatorname{B}(\PL_n)$.

Passing to the direct limit as $n \varrightarrow{} \infty$, there is a canonical map \[\BO \varrightarrow{} \operatorname{B}(\PL).\] Here $\BO$\index{BO(n),BSO(n)@$\BO(n),\BSO(n)$} denotes the stable Grassmann manifold $\varinjlim \BO_n = \varinjlim \grassmannian_n(\bR^\infty)$. According to \cite{hirsch1959} and Mazur, the relative homotopy group \[\pi_k(\operatorname{B}(\PL), \BO)\] is isomorphic to the group $\Gamma_{k-1}$ consisting of all oriented diffeomorphism\index{diffeomorphism} classes of twisted $(k-1)$-spheres (i.e., smooth manifolds obtained by pasting together the boundaries of two closed $(k-1)$-disks). This group is trivial for $k \leq 7$ and is finite for all values of $k$. See \cite{kervaire-milnor} and \cite{cerf}. \defemph{It follows that the rational cohomology $\homology^\ast(\operatorname{B}(\PL);\mathbb{Q})$ is isomorphic to $\homology^\ast(\BO;\mathbb{Q}))$, being a polynomial algebra generated by the Pontrjagin classes.} (Compare Section \ref{ch:20}.) Note however that with integral coefficients, the map \[\homology^\ast(\operatorname{B}(\PL))/\text{torsion} \varrightarrow{} \homology^\ast(\BO)/\text{torsion}\]
is not an e pimorphism. (Compare the integrality condtions in Example 2 of Section \ref{ch:20}.) For the cohomology of $\operatorname{B}(\PL)$ with other coefficients, see \cite{williamson} and \cite{brumfiel}.\index{cohomology!\indexline of $\operatorname{B}\PL$} \index{cohomology!\indexline of $\BO$} 

A fundamental theorem of \cite{hirsch1959},\cite{munkres1964} and \cite{munkres1968} asserts that a piecewise linear manifold $M$ possesses a compatible smoothness structure\index{smoothness structure} if and only if the classifying map \[M \varrightarrow{} \operatorname{B}(\PL) \] for its stable tangent bundle lifts to $\BO$ (compare \cite{milnor1965}), or equivalently if and only if each of a sequence of obstructions\index{obstruction} lying in the groups $\homology^k(M;\Gamma_{k-1})$ is zero.

The theory of topological $\bR^n$-bundles\index{Rn-bundle@$\bR^n$-bundle!\indexline topological} and topological tangent bundles\index{tangent bundle $\tangentbundle{M}$} is completely analogous. In this case the classifying space is denoted by $\operatorname{B}(\operatorname{Top}_n)$. There is a canonical map \[\operatorname{B}(\PL_n) \varrightarrow{} \operatorname{B}(\operatorname{Top}_n).\] In the limit as $n \varrightarrow{} \infty$, an amazing theorem due to \cite{Kirby1969OnTT} asserts that \defemph{the relative homotopy group}\index{homotopy group} \[\pi_k(\operatorname{B}(\operatorname{Top}),\operatorname{B}(\PL))\] \defemph{is zero for $k \neq 4$ and cyclic of order $2$ for $k=4$.} Further they show that a topological manifold\index{topological manifold} $M$ of dimension $\geq 5$ can be triangualted\index{triangulation} as a piecewise linear manifold if and only if the classifying map \[M \varrightarrow{} \operatorname{B}(\operatorname{Top})\] for its stable tangent bundle lifts to $\operatorname{B}(\PL)$, or if and only if a single topological characteristic class in the group \[\homology^4(M;\mathbb{Z}/2)\] is zero.

It follows incidentally that the ring $\homology^\ast(\operatorname{B}(\operatorname{Top});\Lambda)$\index{cohomology!\indexline of $\operatorname{B}(\operatorname{TOP})$} of topological characteristic classes\index{characteristic class} is isomorphic to $\homology^\ast(\operatorname{B}(\PL);\Lambda)$ for any ring $\Lambda$ containing $1/2$. This of course implies Novikov's theorem that rational Pontrjagin classes are topological invariants.

An even broader category of  ``manifolds'' is provided by the class of all \defemphi{Poincaré complexes}: that is, CW-complexes $M$ which satisfy the Poincaré duality theorem\index{Poincar\'e duality} (with arbitrary local coefficients in the non-simply connected case) with respect to some fundamental homology class $\mu \in \homology_n(M;\mathbb{Z})$.\index{fundamental class!\indexline homology}

In order to study such objects, we must introduce a very different type of ``bundle.'' A continuous map $p:E \varrightarrow{} B$ is said to be a \defemphi{fibration} over $B$ or to satisfy the \defemph{covering homotopy property}\index{covering homotopy} if for any space $X$ and any map $f:X \varrightarrow{} E$ any homotopy of $p \circ f$ can be covered by a homotopy of $f$. (Compare \cite{hurewicz}, \cite{dold1963}.) Such a fibration is \defemph{$k$-spherical} if each fiber $p^{-1}(b)$ has the homotopy type of a $k$-sphere.

According to \cite{spivak}, any simply connected Poincaré complex $M$ admits an essentially unique spherical fibration $E \varrightarrow{} M$ with the property that the top homology class in the associated Thom space\index{Thom space} $\thom$ belongs to the image of the Hurewicz homomorphism\index{Hurewicz homomorphism} \[\pi_{n+k+1}(\thom) \varrightarrow{} \homology_{n+k+1}(\thom;\mathbb{Z}).\] More precisely this fibration, called the \defemph{Spivak normal bundle of $M$}\index{Spivak normal bundle}, is unique up to stable fiber homotopy equivalence (which we will not define). 

According to \cite{stasheff1963}, such spherical fibrations over $M$ are classified, up to stable fiber homotopy equivalence, by maps into a classifying space $\operatorname{B}(\operatorname{F})$.\index{BF@$\operatorname{BF}$} There are maps \[\BO \varrightarrow{} \operatorname{B}(\PL) \varrightarrow{} \operatorname{B}(\operatorname{Top}) \varrightarrow{} \operatorname{B}(\operatorname{F}),\]
canonically defined up to homotopy. According to \cite{Browder1968SurgeryAT}, a simply connected Poincaré complex $M$ of formal dimension $n \geq 5$ has the homotopy type of a closed piecewise linear manifold \index{piecewise linear manifold} $M'$ if and only if the classifying map $M \varrightarrow{} \operatorname{B}(\operatorname{F})$ lifts to $\operatorname{B}(\PL)$. (The uniqueness problem for $M'$, studied first by \cite{Novikov_1967} in the differentiable case, is much more complicated.)

The homotopy group $\pi_i(\operatorname{B}(\operatorname{F}))$ is isomorphic to the stable $(i-1)$-stem $\pi_{N-i+1}(S^N)$ for $i \geq 2$ and hence is always finite. The cohomology of this classifying space $\operatorname{B}(\operatorname{F})$ has been studied by \cite{milgram}, \cite{may2006geometry}, and others.

\index{cohomology!\indexline of $\operatorname{BF}$}\index{cohomology!\indexline of $\operatorname{BPL}$}The computations of $\homology^\ast(\operatorname{B}(\PL))$ and $\homology^\ast(\operatorname{B}(\operatorname{F}))$ involve machinery quite differeent from that developed in these notes. Rather than working out these groups from particular characteristic classes, the approaches analyze the homotopy type in terms of associated fibrations or in terms of additional internal structure. \cite{sullivan_2006} for example shows that, ``at odd primes,'' $\BO$ has the homotopy type of the fiber of $\operatorname{B}(\PL) \varrightarrow{} \operatorname{B}(\operatorname{F})$. \cite{boardman1973}, \cite{may2006geometry}, and \cite{SEGAL1974293} have shown that the stable classifying space $\operatorname{B}(\PL)$, $\operatorname{B}(\operatorname{Top})$, and $\operatorname{B}(\operatorname{F})$ all have the homotopy types of infinite loop spaces, so not just the Steenrod algebra but also its homology analogue the Dyer-Lashof algebra\index{Dyer-Lashof algebra} can be brought to bear. Although the Wu classes of Section \ref{ch:19} and their Bocksteins\index{Bockstein homomorphism} play an important role (\cite{milnor1968},\cite{stasheff1968}), other classes appear whose interpretation in terms of fiber space structure or geometry is far from clear \cite{Ravenel1972}.



\end{document}