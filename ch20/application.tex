\documentclass[../main]{subfiles}
\begin{document}
\section{Applications}
We will first discuss an example which was discovered independently by \cite[p, 81]{thom1955-6}, \cite{tamura}, and \cite{shimada}. Two lemmas will be needed. 

\begin{lemma}\label{lem:20.08}\index{Pontrjagin class $\pontrjaginclass_i$}
Let $\xi$ be a smooth vector bundle with projection map $\pi: \total \varrightarrow{} \base$. Then the tangential Pontrjagin class $\pontrjaginclass(\total)=\pontrjaginclass(\tangentbundle{\total})$ of the total space is equal to $\pi^{*}(\pontrjaginclass(\xi) \pontrjaginclass(\tangentbundle{\base}))$, up to $2$-torsion.
\end{lemma}
\begin{proof} Choosing a Riemannian metric on $\total$, the tangent bundle $\tangentbundle{\total}$ clearly splits as the Whitney sum of the bundle of vectors tangent to the fiber and the bundle of vectors normal to the fiber. Since these are isomorphic to $\pi^{*}(\xi)$ and $\pi^{*}(\tangentbundle{B})$ respectively, the conclusion follows.\end{proof}

Let $u \in \homology^{4}(S^{4})$ denote the standard cohomology generator.

\begin{lemma}\label{lem:20.09}
There exists an oriented $4$-plane bundle $\xi^{4}$ over $S^{4}$ with\newline $\pontrjaginclass_{1}(\xi^{4})=-2 u$ and with $\eulerclass(\xi^{4})=u$.\index{Euler class $\eulerclass$}
\end{lemma} 
\begin{proof} Let $\mathbb{H}$\index{quaternions $\mathbb{H}$} denote the non-commutative field of quaternions. (The letter $\mathbb{H}$ honors William Rowan Hamilton.) Then we can form the projective space $\projective^{m}(\mathbb{H})$ of quaternion lines through the origin in $\mathbb{H}^{m+1}$. This is a smooth $4 m$-dimensional manifold. There is a canonical ``quaternion line bundle'' $\tautological$ over $\projective^{m}(\mathbb{H})$\index{projective space!\indexline quaternionic $\projective^n(\mathbb{H})$} whose total space $\total(\tautological)$ is the set of all pairs $(L, v)$ consisting of a quaternion 1-dimensional subspace $L \subset \mathbb{H}^{m+1}$ and a vector $v \in L$. The space of unit vectors in $\total(\tautological)$ can be identified with the unit sphere $S^{4 m+3} \subset \mathbb{H}^{m+1}$.

Using the natural inclusion mappings $\mathbb{R} \subset \mathbb{C} \subset \mathbb{H}$, it follows that there is an underlying complex $2$-plane bundle, which we denote by $\tautological_{\mathbb{C}}$, and an underlying real $4$-plane bundle $\tautological_{\mathbb{R}}$, all over the same base space $\projective^{m}(\mathbb{H})$. From the Gysin sequence\index{Gysin sequence} of $\tautological_{\mathbb{R}}$, we see that the cohomology ring $\homology^{*}(\projective^{m}(\mathbb{H}))$\index{cohomology!\indexline of $\projective^n(\mathbb{H})$} with integer coefficients is a truncated polynomial ring, generated by the Euler or Chern class
\[
\eulerclass(\tautological_{\mathbb{R}})=\chernclass_{2}(\tautological_{\mathbb{C}}) \in \homology^{4}(\projective^{m}(\mathbb{H}))
\]\index{Chern class $\chernclass_i$}
Denoting this cohomology generator briefly by $u \in \homology^{4}(\projective^{m}(\homology))$, it follows that the total Chern class is given by
\[
\chernclass(\tautological_{\mathbb{C}})=1+u ,
\]
hence the total Pontrjagin class is
\[
\pontrjaginclass(\tautological_{\mathbb{R}})=(1-u)^{2}=1-2 u+u^{2},
\]
by \ref{cor:15.05}. Now specializing to the quaternion projective line $\projective^{1}(\mathbb{H}) \cong S^{4}$, we have
\[
\pontrjaginclass_{1}(\tautological_{\mathbb{R}})=-2 u, \quad \eulerclass(\tautological_{\mathbb{R}})=u
\]
as required.
\end{proof}

For any even integer $k$, it follows that there exists a bundle $\xi$ over $S^{4}$ with $\pontrjaginclass_{1}(\xi)=ku$. One can simply take $\xi=f^{*}(\tautological_{\mathbb{R}})$ where $f: S^{4} \varrightarrow{} S^{4}$ is a map of degree $-k / 2$. This is a best possible result, since $\pontrjaginclass_{1}(\xi)$ cannot be an odd multiple of u by Problem \ref{prob:15-A}.

(For vector bundles over the sphere $S^{4 m}$ the corresponding best possible result is that the Pontrjagin class $\pontrjaginclass_{m}(\xi)$ can be any multiple of \newline $(2m-1)!\mathrm{GCD}(m+1,2) u$. The proof of this statement is based on the Bott periodicity\index{Bott periodicity} theorem. Compare \cite{bott1970}
\setcounter{example}{0}

\begin{example}
Let $\xi^{n}$ be a smooth n-plane bundle over the sphere $S^{4}$. For convenience, we assume that $n \geq 5$. Choosing a Euclidean metric\index{Euclidean metric}, let $\total^{\prime} \subset \total(\xi^{n})$ be the set of vectors of length $\leq 1$, and let $\partial \total^{\prime}$ be the set of vectors of length precisely $1 .$
\end{example} 

Using the remarks above, we see that $\pontrjaginclass_{1}(\xi^{n})=ku$ where $k$ can be an arbitrary even integer. Hence
\[
\pontrjaginclass_{1}(\total(\xi^{n}))=k \pi^{*}(u)
\]
by \ref{lem:20.08}. Since $\partial \total^{\prime}$ has trivial normal bundle in $\total(\xi^{n})$, it follows that
\[
\pontrjaginclass_{1}(\partial \total^{\prime})=ku^{\prime}
\]
where $u^{\prime} \epsilon \homology^{4}(\partial \total^{\prime})$ is the standard generator which corresponds to $u$ under the homomorphism
\[
\homology^{4}(S^{4}) \varrightarrow{} \homology^{4}(\partial \total^{\prime})
\]
from the Gysin sequence of $\xi^{n}$.\index{Gysin sequence}

Since the Pontrjagin class $\pontrjaginclass_{1}$ of the smooth manifold $\partial \total^{\prime}$ is a combinatorial invariant, it follows that the even integer $|k|$ is also a combinatorial invariant. \defemph{Thus as $k$ varies we obtain infinitely many smooth manifolds $\partial \total^{\prime}$ of fixed dimension $n+3 \geq 8$ which are combinatorially distinct.}

On the other hand, according to \cite{james1954}, these manifolds $\partial \total^{\prime}$ for fixed $n$ fall into a finite number (namely $13$) of distinct homotopy types. \defemph{Thus for any fixed dimension $\geq 8$ there must exist two smooth simply-connected manifolds which have the same homotopy type but are not piecewise linearly homeomorphic.} (The dimension $8$ can easily be improved to $7$.)\index{homotopy type}

Using Novikov's theorem that rational Pontrjagin classes are topological invariants, it follows of course that these manifolds are not even homeomorphic.

A quite different example of manifolds which have the same homotopy type but are not homeomorphic involves the study of the fundamental group, for example of a 3-dimensional lens space.\index{lens space} (See \cite{brody} and \cite{chapman})

The next example is due to \cite{thom1968}. (See also \cite{milnor1956} and \cite{shimada}.) We must first sharpen \ref{lem:20.09}

\begin{lemma}\label{lem:20.10} Given integers $k, l$ satisfying $k \equiv 2 l\pmod 4$, there exists an oriented $4$-plane bundle $\xi$ over $S^{4}$ with $\pontrjaginclass_{1}(\xi)=ku,\, \eulerclass(\xi)=l u$

\end{lemma}

(These integers $k$ and $l$ actually determine the isomorphism class of the bundle $\xi$, since the homotopy group $\pi_{4}(\xtilde{\grassmannian}_{4}) \cong \pi_{3}(\SO_{4})$ is isomorphic to $\mathbb{Z} \oplus \mathbb{Z}$.)\index{Grassmannian manifold!\indexline oriented $\xtilde{\grassmannian}_n$}\index{homtopy groups}

\begin{proof} Recall that the space of oriented $4$-planes in $\mathbb{R}^{\infty}$ is denoted by $\xtilde{\grassmannian}_{4}$. For every homotopy class $(f)$ in the homotopy group $\pi_{4}(\xtilde{\grassmannian}_{4})$ we can form the cohomology class
\[
\pontrjaginclass_{1}(f^{*} \xtilde{y}^{4})=f^{*} \pontrjaginclass_{1}(\xtilde{y}^{4})
\]
in the group $\homology^{4}(S^{4})$ with integer coefficients by pulling the universal bundle $\xtilde{\tautological}^{4}$ back to the $4$-sphere and then taking its Pontrjagin class. This correspondence $(f) \mapsto \pontrjaginclass_{1}(f^{*} \xtilde{\tautological}^{4})$ from $\pi_{4}(\xtilde{\grassmannian}_{4})$ to $\homology^{4}(S^{4}) \cong \mathbb{Z}$ is an additive homomorphism, as one sees by noting that
\[
\langle\pontrjaginclass_{1}(f^{*} \xtilde{\tautological}^{4}), \mu_{4}\rangle=\langle\pontrjaginclass_{1}(\xtilde{\tautological}^{4}), f_{*}(\mu_{4})\rangle
\]
where the Hurewicz homomorphism\index{Hurewicz homomorphism}
\[
(f)  \mapsto f_{*}(\mu_{4})
\]
is well known to be a homomorphism. Similarly the Euler class gives rise to an additive homomorphism
\[
(f) \mapsto \eulerclass(f^{*} \xtilde{\tautological}^{4})
\]
from $\pi_{4}(\xtilde{\grassmannian}_{4})$ to $\homology^{4}(S^{4}) \cong \mathbb{Z}$.

Now the tangent bundle of $S^{4}$ is isomorphic to $f_{1}^{*} \xtilde{y}^{4}$, and the bundle $\tautological_{\mathbb{R}}$ of \ref{lem:20.09} is isomorphic to $f_{2}^{*} \xtilde{\tautological}^{4}$ for suitable maps $f_{1}, f_{2}: S^{4} \varrightarrow{} \xtilde{\grassmannian}_{4}$. Thus
\[
\begin{aligned}
&\pontrjaginclass_{1}(f_{1}^{*} \xtilde{\tautological}^{4})=0, & \eulerclass(f_{1}^{*} \xtilde{\tautological}^{4})=2 u& \\
&\pontrjaginclass_{1}(f_{2}^{*} \xtilde{\tautological}^{4})=-2 u, & \eulerclass(f_{2}^{*} \xtilde{\tautological}^{4})=u&
\end{aligned}
\]
Taking a suitable linear combination $(f)$ of $(f_{1})$ and $(f_{2})$, we can now clearly obtain
\[
\pontrjaginclass_{1}(f^{*} \xtilde{\tautological}^{4})=ku, \quad \eulerclass(f^{*} \xtilde{\tautological}^{4})=lu
\]
for any integers $k$ and $l$ satisfying $k \equiv 2 l\pmod 4$. 

\end{proof}

\begin{example} For any integer $k \equiv 2\pmod 4$, there exists by \ref{lem:20.10} an oriented $4$-plane bundle $\xi$ over $S^{4}$ with
\[
\pontrjaginclass_{1}(\xi)=ku, \quad \eulerclass(\xi)=u .
\]\index{Pontrjagin number}
Using the Gysin sequence\index{Gysin sequence} of $\xi$, it follows easily that the space $\partial \total^{\prime}$ of unit vectors in $\total(\xi)$ has the homotopy type of the sphere $S^{7}$. In fact this manifold $\partial \total^{\prime}$ is actually homeomorphic to the $7$-sphere. As a smooth manifold, it can be obtained by identifying the boundaries of two copies of the unit $7$-disk by a suitable (but possibly exotic) diffeomorphism between their boundary $6$-spheres. This fact is proved directly in \cite{milnor1956}, and is also a consequence of the Generalized Poincar\'e Hypothesis\index{Poincar\'e Hypothesis} as proved by \cite{smale1959}. Now starting with a smooth triangulation of the $6$-sphere and then extending to smooth triangulations of the two $7$-disks, it follows easily that \defemph{the manifold $\partial \total^{\prime}$ is even combinatorially equivalent to the $7$-sphere}.
\end{example}

Consider the Thom space $\thom=\thom(\xi)$. Evidently $\thom$ can be identified with the manifold obtained from $\total^{\prime}$ by adjoining a cone over $\partial \total^{\prime}$. Choosing a smooth triangulation of $\total^{\prime}$, since $\partial \total^{\prime}$ is a combinatorial sphere, \defemph{it follows that $\thom=\thom(\xi)$ can be triangulated as a piecewise linear manifold}\index{piecewise linear manifold}\index{triangulation}. That is it can be triangulated so that every point of $\thom$ has a neighborhood piecewise linearly homeomorphic to $\mathbb{R}^{8}$.

According to \ref{lem:18.1} or \ref{lem:18.2}, the homology groups of $\thom$ are infinite cyclic in dimensions $0,4,8$, and zero otherwise. Thus the signature $\sigma(\thom)$\index{signature $\sigma$} must be $\pm 1$, and choosing the orientation correctly we may assume that $\sigma(\thom)=+1$.

By \ref{lem:20.08} the tangential Pontrjagin class $\pontrjaginclass_{1}(\total^{\prime})$ is $k$ times a cohomology generator. Hence $\pontrjaginclass_{1}(\thom)$ is $k$ times a generator, and the Pontrjagin number $\pontrjaginclass_{1}^{2}[\thom]$ must be equal to $k^{2}$. Using the signature theorem
\[
\sigma(\thom)=\frac{7}{45} \pontrjaginclass_{2}[\thom]-\frac{1}{45} \pontrjaginclass_{1}^{2}[\thom],
\]
it follows that the other Pontrjagin number is given by
\[
\pontrjaginclass_{2}[\thom]=\dfrac{45+k^{2}}{7} .
\]
Here $k$ can be any integer congruent to $2$ modulo $4$. But if $k \equiv \pm 2\pmod{7}$ this is not an integer. (For example if $k=6$, then $\pontrjaginclass_{2}[\thom]$ is not an integer.) Since the Pontrjagin numbers of a smooth manifold must be integers, we have proved the following assertion.

\defemph{For $k \equiv \pm 2\pmod 7$, the triangulated 8-dimensional manifold $\thom=\thom(\xi)$ possesses no smoothness structure which is compatible with the given triangulation.}\index{smoothness structure}

As a corollary, it follows that \defemph{the smooth $7$-dimensional manifold $\partial \total^{\prime}$ (which is homeomorphic to $S^{7}$) is not diffeomorphic to $S^{7}$}. For otherwise $\thom$ could clearly be given a compatible smoothness structure.\index{diffeomorphism}

We conclude with a problem for the reader.

\begin{problem}\label{prob:20.A} Let $\tangentbundle{}$ be the tangent bundle of the quaternion projective space $\projective^{m}(\mathbb{H})$\index{projective space!\indexline quaternionic $\projective^n(\mathbb{H})$}\index{quaternions $\mathbb{H}$}. (See the proof of Lemma \ref{lem:20.09}.) Using the isomorphism $\tangentbundle{} \cong \Hom_{\mathbb{H}}(\tautological, \tautological^{\perp})$ of real vector bundles show that\index{Hom}
\[
\tangentbundle{} \oplus \Hom_{\mathbb{H}}(\tautological, \tautological) \cong \Hom_{\mathbb{H}}(\tautological, \mathbb{H}^{m+1}),
\]
and hence that $\pontrjaginclass(\tangentbundle{})=\dfrac{(1+u)^{2 m+2} }{1+4 u}$. (Compare \cite{szczarba} as well as Section \ref{thm:14.10})

\end{problem}

\end{document}